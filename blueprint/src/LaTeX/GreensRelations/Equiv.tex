% This file is included by both the web and print versions of the blueprint
% via an import statment in blueprint/scr/content.tex 

\section{Green's Equivalences (\(\mathcal R, \mathcal L, \mathcal J, \mathcal H\))}

\begin{definition}[Right equivalence]
\label{def:REquiv}
Let \(M\) be a monoid.  For \(x,y \in M\) we define \(x \,\mathcal R\, y\) to hold if both \(x \le_{\mathcal R} y\) and \(y \le_{\mathcal R} x\) hold.  In other words, two elements are right equivalent when they lie in the principal right ideals generated by each other.
\lean{REquiv}
\leanok
\end{definition}

\begin{lemma}[Right equivalence is an equivalence relation]
\label{lem:REquiv-isEquivalence}
The relation \(\mathcal R\) defined in Definition~\ref{def:REquiv} is an equivalence relation on \(M\).
\lean{REquiv.isEquivalence}
\leanok
\uses{def:REquiv, def:EquivOfLE, lem:EquivOfLE-isEquivalence, lem:RRel-isPreorder}
\end{lemma}

\begin{definition}[Left equivalence]
\label{def:LEquiv}
For \(x,y\in M\) we define \(x \,\mathcal L\, y\) to hold when both \(x \le_{\mathcal L} y\) and \(y \le_{\mathcal L} x\) hold.  This expresses that \(x\) and \(y\) generate the same principal left ideals.
\lean{LEquiv}
\leanok
\end{definition}

\begin{lemma}[Left equivalence is an equivalence relation]
\label{lem:LEquiv-isEquivalence}
The relation \(\mathcal L\) defined in Definition~\ref{def:LEquiv} is an equivalence relation on \(M\).
\lean{LEquiv.isEquivalence}
\leanok
\uses{def:LEquiv, def:EquivOfLE, lem:EquivOfLE-isEquivalence, lem:LRel-isPreorder}
\end{lemma}

\begin{definition}[J equivalence]
\label{def:JEquiv}
For \(x,y\in M\) we define \(x \,\mathcal J\, y\) to hold when both \(x \le_{\mathcal J} y\) and \(y \le_{\mathcal J} x\) hold, i.e., each lies in the two‑sided ideal generated by the other.
\lean{JEquiv}
\leanok
\end{definition}

\begin{lemma}[J equivalence is an equivalence relation]
\label{lem:JEquiv-isEquivalence}
The relation \(\mathcal J\) defined in Definition~\ref{def:JEquiv} is an equivalence relation on \(M\).
\lean{JEquiv.isEquivalence}
\leanok
\uses{def:JEquiv, def:EquivOfLE, lem:EquivOfLE-isEquivalence, lem:JRel-isPreorder}
\end{lemma}

\begin{definition}[H equivalence]
\label{def:HEquiv}
For \(x,y\in M\) we define \(x \,\mathcal H\, y\) to hold when both \(x \le_{\mathcal H} y\) and \(y \le_{\mathcal H} x\) hold.  This captures when two elements are simultaneously right and left equivalent.
\lean{HEquiv}
\leanok
\end{definition}

\begin{lemma}[H equivalence is an equivalence relation]
\label{lem:HEquiv-isEquivalence}
The relation \(\mathcal H\) defined in Definition~\ref{def:HEquiv} is an equivalence relation on \(M\).
\lean{HEquiv.isEquivalence}
\leanok
\uses{def:HEquiv, def:EquivOfLE, lem:EquivOfLE-isEquivalence, lem:HRel-isPreorder}
\end{lemma}
