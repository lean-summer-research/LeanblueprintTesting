\section{Equivalence classes}

Green's equivalence relations partition a monoid into subsets called equivalence classes.  Concretely, given an element \(a\in M\), the \(\mathcal R\)-class of \(a\) is the set of all \(x\) such that \(x \,\mathcal R\, a\).  Similarly for \(\mathcal L, \mathcal J, \mathcal H\) and \(\mathcal D\).

\begin{definition}[Right class]
\label{def:RClass}
For \(a\in M\) we define \(\mathrm{RClass}(a)\) to be the set \(\{x\in M \mid x \,\mathcal R\, a\}\).
\lean{RClass}
\leanok
\uses{def:REquiv}
\end{definition}

\begin{definition}[Left class]
\label{def:LClass}
For \(a\in M\) we define \(\mathrm{LClass}(a)\) to be the set \(\{x\in M \mid x \,\mathcal L\, a\}\).
\lean{LClass}
\leanok
\uses{def:LEquiv}
\end{definition}

\begin{definition}[J class]
\label{def:JClass}
For \(a\in M\) we define \(\mathrm{JClass}(a)\) to be the set \(\{x\in M \mid x \,\mathcal J\, a\}\).
\lean{JClass}
\leanok
\uses{def:JEquiv}
\end{definition}

\begin{definition}[H class]
\label{def:HClass}
For \(a\in M\) we define \(\mathrm{HClass}(a)\) to be the set \(\{x\in M \mid x \,\mathcal H\, a\}\).
\lean{HClass}
\leanok
\uses{def:HEquiv}
\end{definition}

\begin{definition}[D class]
\label{def:DClass}
For \(a\in M\) we define \(\mathrm{DClass}(a)\) to be the set \(\{x\in M \mid x \,\mathcal D\, a\}\).
\lean{DClass}
\leanok
\uses{def:DEquiv}
\end{definition}
