\section{R--L multiplicative compatibility}

\begin{lemma}[Left multiplication respects the right preorder]
\label{lem:RRel-mul_left_compat}
For all \(x,y,z \in M\), if \(x \le_R y\) then \(z\cdot x \le_R z\cdot y\).
\lean{RRel.mul_left_compat}
\leanok
\uses{def:RRel}
\end{lemma}
\begin{proof}
\leanok
Suppose \(x \le_R y\), so there exists \(u\in M\) with \(y \cdot u = x\).  We want to show \(z\cdot x \le_R z\cdot y\), meaning there is a witness \(u'\) such that \((z\cdot y)\cdot u' = z\cdot x\).  Taking \(u' = u\) we compute \((z\cdot y)\cdot u' = z\cdot(y\cdot u) = z\cdot x\) by associativity.  Hence \(z\cdot x \le_R z\cdot y\).
\uses{def:RRel}
\end{proof}

\begin{lemma}[Right multiplication respects the left preorder]
\label{lem:LRel-mul_right_compat}
For all \(x,y,z \in M\), if \(x \le_L y\) then \(x\cdot z \le_L y\cdot z\).
\lean{LRel.mul_right_compat}
\leanok
\uses{def:LRel}
\end{lemma}
\begin{proof}
\leanok
Assume \(x \le_L y\), so there exists \(u \in M\) with \(u\cdot y = x\).  To prove \(x\cdot z \le_L y\cdot z\) we need a witness \(u'\) with \(u'\cdot (y\cdot z) = x\cdot z\).  Taking \(u'=u\) and using associativity gives \(u'\cdot (y\cdot z) = u\cdot (y\cdot z) = (u\cdot y)\cdot z = x\cdot z\).  Thus \(x\cdot z \le_L y\cdot z\).
\uses{def:LRel}
\end{proof}

\begin{lemma}[Left multiplication respects right equivalence]
\label{lem:REquiv-mul_left_compat}
For all \(x,y,z\in M\), if \(x\) and \(y\) are right equivalent then \(z\cdot x\) and \(z\cdot y\) are right equivalent.
\lean{REquiv.mul_left_compat}
\leanok
\uses{def:REquiv, lem:RRel-mul_left_compat}
\end{lemma}
\begin{proof}
\leanok
Suppose \(x\) and \(y\) are right equivalent, meaning \(x \le_R y\) and \(y \le_R x\).  By Lemma~\ref{lem:RRel-mul_left_compat}, left multiplication by \(z\) preserves the right preorder; hence \(z\cdot x \le_R z\cdot y\) and \(z\cdot y \le_R z\cdot x\).  Therefore \(z\cdot x\) and \(z\cdot y\) are right equivalent.
\uses{def:REquiv, lem:RRel-mul_left_compat}
\end{proof}

\begin{lemma}[Right multiplication respects left equivalence]
\label{lem:LEquiv-mul_right_compat}
For all \(x,y,z\in M\), if \(x\) and \(y\) are left equivalent then \(x\cdot z\) and \(y\cdot z\) are left equivalent.
\lean{LEquiv.mul_right_compat}
\leanok
\uses{def:LEquiv, lem:LRel-mul_right_compat}
\end{lemma}
\begin{proof}
\leanok
Assume \(x\) and \(y\) are left equivalent, so \(x \le_L y\) and \(y \le_L x\).  By Lemma~\ref{lem:LRel-mul_right_compat}, right multiplication by \(z\) preserves the left preorder, giving \(x\cdot z \le_L y\cdot z\) and \(y\cdot z \le_L x\cdot z\).  Hence \(x\cdot z\) and \(y\cdot z\) are left equivalent.
\uses{def:LEquiv, lem:LRel-mul_right_compat}
\end{proof}
