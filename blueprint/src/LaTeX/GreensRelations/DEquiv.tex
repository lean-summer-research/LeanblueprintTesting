\section{The D-equivalence}

\begin{definition}[D-equivalence]
\label{def:DEquiv}
Let \(M\) be a monoid. For \(x,y\in M\) we say \(x \,\mathcal D\, y\) holds if there exists \(z\in M\) such that
\(x \,\mathcal R\, z\) and \(z \,\mathcal L\, y\).
Equivalently, there is a chain \(x \,\mathcal R\, z\) and \(z \,\mathcal L\, y\).
\lean{DEquiv}
\leanok
\end{definition}


\begin{lemma}[D-equivalence symmetry]
\label{lem:DEquiv-comm-symm}
For all \(x,y\in M\),
\[
 x \,\mathcal D\, y
\;\Longleftrightarrow\;
 (\exists z,\; x \,\mathcal R\, z \land z \,\mathcal L\, y)
\;\Longleftrightarrow\;
 (\exists w,\; y \,\mathcal R\, w \land w \,\mathcal L\, x),
\]
and in particular \(x \,\mathcal D\, y \Rightarrow y \,\mathcal D\, x\).
\lean{DEquiv.comm, DEquiv.symm}
\leanok
\uses{def:DEquiv, def:REquiv, def:LEquiv, lem:rEquiv_lEquiv_comm}
\end{lemma}
\begin{proof}
\leanok
\begin{itemize}
  \item The first equivalence is just Definition~\ref{def:DEquiv} unfolded.
  \item The second equivalence uses the commutation of \(\mathcal R\)- and \(\mathcal L\)-equivalences:
        any chain \(x \,\mathcal R\, z \,\mathcal L\, y\) can be rotated to a chain
        \(y \,\mathcal R\, w \,\mathcal L\, x\) (Lemma \texttt{rEquiv\_lEquiv\_comm}).
  \item Taking \(x,y\) swapped in the second characterization yields the symmetry
        \(x \,\mathcal D\, y \Rightarrow y \,\mathcal D\, x\).
\end{itemize}
\end{proof}


\begin{lemma}[Closure under \(\mathcal L\)- and \(\mathcal R\)-equivalence]
\label{lem:DEquiv-closures}
Let \(x,y,z\in M\).
\begin{itemize}
  \item If \(x \,\mathcal D\, y\) and \(y \,\mathcal L\, z\), then \(x \,\mathcal D\, z\).
  \item If \(x \,\mathcal D\, y\) and \(y \,\mathcal R\, z\), then \(x \,\mathcal D\, z\). \hfill(\emph{by duality from the previous bullet, using symmetry})
\end{itemize}
\lean{DEquiv.closed_under_lEquiv, DEquiv.closed_under_rEquiv}
\leanok
\uses{def:DEquiv, def:LEquiv, def:REquiv, lem:LEquiv-isEquivalence, lem:DEquiv-comm-symm}
\end{lemma}
\begin{proof}
\leanok
\begin{itemize}
  \item Write \(x \,\mathcal D\, y\) as \(x \,\mathcal R\, u\) and \(u \,\mathcal L\, y\) for some \(u\).
        If \(y \,\mathcal L\, z\), then by transitivity of \(\mathcal L\)-equivalence,
        \(u \,\mathcal L\, z\). Hence \(x \,\mathcal R\, u \,\mathcal L\, z\), i.e. \(x \,\mathcal D\, z\).
  \item For the \(\mathcal R\)-closure: from \(x \,\mathcal D\, y\) get \(y \,\mathcal D\, x\) by symmetry
        (Lemma~\ref{lem:DEquiv-comm-symm}); combine with \(y \,\mathcal R\, z\) and apply the previous bullet
        in the dual form to obtain \(y \,\mathcal D\, z\), then symmetrize back to \(x \,\mathcal D\, z\).
\end{itemize}
\end{proof}

\begin{lemma}[Transitivity of \(\mathcal D\)]
\label{lem:DEquiv-trans}
If \(x \,\mathcal D\, y\) and \(y \,\mathcal D\, z\) then \(x \,\mathcal D\, z\).
\lean{DEquiv.trans}
\leanok
\uses{def:DEquiv, lem:DEquiv-closures}
\end{lemma}
\begin{proof}
\leanok
Choose witnesses \(x \,\mathcal R\, u \,\mathcal L\, y\) and \(y \,\mathcal R\, v \,\mathcal L\, z\).
Apply the \(\mathcal R\)-closure to get \(x \,\mathcal D\, v\), then the \(\mathcal L\)-closure to conclude \(x \,\mathcal D\, z\).
\end{proof}

\begin{lemma}[\(\mathcal D\) is an equivalence relation]
\label{lem:DEquiv-isEquiv}
The relation \(\mathcal D\) on \(M\) is reflexive, symmetric, and transitive.
\lean{DEquiv.isEquiv}
\leanok
\uses{def:DEquiv, lem:DEquiv-comm-symm, lem:DEquiv-trans}
\end{lemma}
\begin{proof}
\leanok
\begin{itemize}
  \item \textbf{Reflexive:} take \(z:=x\) and use reflexivity of \(\mathcal R\) and \(\mathcal L\).
  \item \textbf{Symmetric:} Lemma~\ref{lem:DEquiv-comm-symm}.
  \item \textbf{Transitive:} Lemma~\ref{lem:DEquiv-trans}.
\end{itemize}
\end{proof}