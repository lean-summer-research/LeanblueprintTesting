\section{The D-equivalence}

\begin{definition}[D-equivalence]
\label{def:DEquiv}
Let \(M\) be a monoid.  For \(x,y\in M\) we say \(\mathrm{DEquiv}\,x\,y\) holds if there exists an element \(z\in M\) such that \(x\) is right equivalent to \(z\) and \(z\) is left equivalent to \(y\).  In other words, there is a chain \(x \mathrel{REquiv} z\) and \(z \mathrel{LEquiv} y\).  This relation is Green's \(D\)-equivalence.
\lean{DEquiv}
\leanok
\end{definition}

\begin{lemma}[Characterization of \(\mathrm{DEquiv}\)]
\label{lem:DEquiv-comm}
For all \(x,y\in M\) we have
\[
  (\exists z\in M,\; x \mathrel{REquiv} z \land z \mathrel{LEquiv} y)\quad\Longleftrightarrow\quad \mathrm{DEquiv}\,x\,y.
\]
\lean{DEquiv.comm}
\leanok
\uses{def:DEquiv, def:REquiv, def:LEquiv, lem:rEquiv_lEquiv_comm}
\end{lemma}
\begin{proof}
\leanok
Unwinding Definition~\ref{def:DEquiv}, \(\mathrm{DEquiv}\,x\,y\) holds exactly when there is a \(z\) with \(x\) right equivalent to \(z\) and \(z\) left equivalent to \(y\).  This is precisely the left‑hand statement.  The commutativity Lemma~\ref{lem:rEquiv_lEquiv_comm} shows that any such chain can be rearranged to start with a right equivalence and end with a left equivalence, so no generality is lost.
\uses{def:DEquiv, lem:rEquiv_lEquiv_comm}
\end{proof}

\begin{lemma}[Reflexivity of \(\mathrm{DEquiv}\)]
\label{lem:DEquiv-refl}
For every \(x\in M\), one has \(\mathrm{DEquiv}\,x\,x\).
\lean{DEquiv.refl}
\leanok
\uses{def:DEquiv, def:REquiv, def:LEquiv}
\end{lemma}
\begin{proof}
\leanok
Fix \(x\in M\).  Taking \(z=x\), reflexivity of the right equivalence (Lemma~\ref{lem:REquiv-isEquivalence}) and the left equivalence (Lemma~\ref{lem:LEquiv-isEquivalence}) implies \(x \sim_R x\) and \(x \sim_L x\).  Hence \(\mathrm{DEquiv}\,x\,x\) by Definition~\ref{def:DEquiv}.
\uses{def:DEquiv, def:REquiv, def:LEquiv, lem:REquiv-isEquivalence, lem:LEquiv-isEquivalence}
\end{proof}

\begin{lemma}[Symmetry of \(\mathrm{DEquiv}\)]
\label{lem:DEquiv-symm}
If \(\mathrm{DEquiv}\,x\,y\) holds then \(\mathrm{DEquiv}\,y\,x\) holds.
\lean{DEquiv.symm}
\leanok
\uses{def:DEquiv, lem:rEquiv_lEquiv_comm}
\end{lemma}
\begin{proof}
\leanok
Assume \(\mathrm{DEquiv}\,x\,y\).  Then there is \(z\) with \(x \sim_R z\) and \(z \sim_L y\).  By Lemma~\ref{lem:rEquiv_lEquiv_comm} there exists an element \(w\) such that \(z \sim_R y\) and \(y \sim_L w\).  Reversing the roles of left and right in that lemma yields \(y \sim_R w\) and \(w \sim_L x\).  Hence by Definition~\ref{def:DEquiv} we have \(\mathrm{DEquiv}\,y\,x\).
\uses{def:DEquiv, lem:rEquiv_lEquiv_comm}
\end{proof}

\begin{lemma}[Closure under left equivalence]
\label{lem:DEquiv-closed_under_lEquiv}
If \(\mathrm{DEquiv}\,x\,y\) and \(y \mathrel{LEquiv} z\), then \(\mathrm{DEquiv}\,x\,z\).
\lean{DEquiv.closed_under_lEquiv}
\leanok
\uses{def:DEquiv, def:LEquiv, lem:LEquiv-isEquivalence}
\end{lemma}
\begin{proof}
\leanok
Suppose \(\mathrm{DEquiv}\,x\,y\).  So there exists \(u\in M\) with \(x \sim_R u\) and \(u \sim_L y\).  If \(y \sim_L z\), then by transitivity of the left equivalence (Lemma~\ref{lem:LEquiv-isEquivalence}) we have \(u \sim_L z\).  Combining \(x \sim_R u\) and \(u \sim_L z\) shows that \(\mathrm{DEquiv}\,x\,z\).
\uses{def:DEquiv, lem:LEquiv-isEquivalence}
\end{proof}

\begin{lemma}[Closure under right equivalence]
\label{lem:DEquiv-closed_under_rEquiv}
If \(\mathrm{DEquiv}\,x\,y\) and \(y \mathrel{REquiv} z\), then \(\mathrm{DEquiv}\,x\,z\).
\lean{DEquiv.closed_under_rEquiv}
\leanok
\uses{def:DEquiv, def:REquiv, lem:rEquiv_lEquiv_comm, lem:DEquiv-symm, lem:DEquiv-closed_under_lEquiv}
\end{lemma}
\begin{proof}
\leanok
Assume \(\mathrm{DEquiv}\,x\,y\) and \(y \sim_R z\).  By Lemma~\ref{lem:DEquiv-symm} we know \(\mathrm{DEquiv}\,y\,x\).  Using commutativity of right and left equivalence (Lemma~\ref{lem:rEquiv_lEquiv_comm}), the right equivalence \(y \sim_R z\) may be interpreted as a left equivalence between suitable elements.  Applying closure under left equivalence (Lemma~\ref{lem:DEquiv-closed_under_lEquiv}) to \(\mathrm{DEquiv}\,y\,x\) and this left equivalence yields \(\mathrm{DEquiv}\,y\,z\).  A final application of Lemma~\ref{lem:DEquiv-symm} gives \(\mathrm{DEquiv}\,x\,z\).
\uses{lem:DEquiv-symm, lem:DEquiv-closed_under_lEquiv, lem:rEquiv_lEquiv_comm}
\end{proof}

\begin{lemma}[Transitivity of \(\mathrm{DEquiv}\)]
\label{lem:DEquiv-trans}
If \(\mathrm{DEquiv}\,x\,y\) and \(\mathrm{DEquiv}\,y\,z\) then \(\mathrm{DEquiv}\,x\,z\).
\lean{DEquiv.trans}
\leanok
\uses{def:DEquiv, lem:DEquiv-closed_under_rEquiv, lem:DEquiv-closed_under_lEquiv}
\end{lemma}
\begin{proof}
\leanok
Assume \(\mathrm{DEquiv}\,x\,y\) and \(\mathrm{DEquiv}\,y\,z\).  By Definition~\ref{def:DEquiv} there exists \(u\) with \(x \sim_R u\) and \(u \sim_L y\), and there exists \(v\) with \(y \sim_R v\) and \(v \sim_L z\).  Applying Lemma~\ref{lem:DEquiv-closed_under_rEquiv} to \(\mathrm{DEquiv}\,x\,y\) and the right equivalence \(y \sim_R v\) gives \(\mathrm{DEquiv}\,x\,v\).  Then Lemma~\ref{lem:DEquiv-closed_under_lEquiv} applied to \(\mathrm{DEquiv}\,x\,v\) and the left equivalence \(v \sim_L z\) yields \(\mathrm{DEquiv}\,x\,z\).
\uses{def:DEquiv, lem:DEquiv-closed_under_rEquiv, lem:DEquiv-closed_under_lEquiv}
\end{proof}

\begin{lemma}[D-equivalence is an equivalence relation]
\label{lem:DEquiv-isEquiv}
The relation defined in Definition~\ref{def:DEquiv} is an equivalence relation on \(M\).
\lean{DEquiv.isEquiv}
\leanok
\uses{def:DEquiv, lem:DEquiv-refl, lem:DEquiv-symm, lem:DEquiv-trans}
\end{lemma}
\begin{proof}
\leanok
An equivalence relation is one that is reflexive, symmetric and transitive.  These three properties are given by Lemmas~\ref{lem:DEquiv-refl}, \ref{lem:DEquiv-symm} and \ref{lem:DEquiv-trans}.  Hence \(\mathrm{DEquiv}\) is an equivalence relation.
\uses{def:DEquiv, lem:DEquiv-refl, lem:DEquiv-symm, lem:DEquiv-trans}
\end{proof}
