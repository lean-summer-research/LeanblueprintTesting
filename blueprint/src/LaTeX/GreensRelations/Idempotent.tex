\section{Preorder Idempotent Properties}

\begin{lemma}[Idempotent characterization for the right preorder]
\label{lem:RRel-idempotent_iff}
Let \(M\) be a monoid and let \(e\in M\) be idempotent (so \(e \cdot e = e\)).  For any \(x\in M\),
\[
  x \le_R e \quad\text{if and only if}\quad e\cdot x = x.
\]
% Lean declaration
\lean{RRel.idempotent_iff}
\leanok
\uses{def:RRel}
\end{lemma}
\begin{proof}
\leanok
Assume \(e\) is idempotent.  First suppose \(x \le_R e\).  Then there exists \(t\) such that \(e \cdot t = x\).  Multiplying the equation \(e \cdot t = x\) on the left by \(e\) and using idempotency yields
\(e \cdot x = e \cdot (e \cdot t) = (e \cdot e) \cdot t = e \cdot t = x\).  Conversely, suppose \(e \cdot x = x\).  Setting \(t=x\) gives \(e \cdot t = e \cdot x = x\), so \(x \le_R e\).
\uses{def:RRel}
\end{proof}

\begin{lemma}[Idempotent characterization for the left preorder]
\label{lem:LRel-idempotent_iff}
Let \(M\) be a monoid and let \(e\in M\) be idempotent.  For any \(x\in M\),
\[
  x \le_L e \quad\text{if and only if}\quad x\cdot e = x.
\]
\lean{LRel.idempotent_iff}
\leanok
\uses{def:LRel}
\end{lemma}
\begin{proof}
\leanok
Assume \(e\) is idempotent.  Suppose \(x \le_L e\).  Then there exists \(t\) such that \(t \cdot e = x\).  Multiplying this equality on the right by \(e\) and using idempotency gives \(x \cdot e = (t \cdot e) \cdot e = t \cdot (e \cdot e) = t \cdot e = x\).  Conversely, suppose \(x \cdot e = x\).  Taking \(t = x\) yields \(t \cdot e = x\cdot e = x\), so \(x \le_L e\).
\uses{def:LRel}
\end{proof}
