\section{Green’s Lemma}

\begin{lemma}[Translation identity for \(\mathcal L\)-below elements]
\label{lem:id-translation}
Let \(M\) be a monoid and suppose \(x,y\in M\) satisfy \(x \,\mathcal R\, y\).  
Choose \(u,v\in M\) with \(x\cdot u = y\) and \(y\cdot v = x\).
Then for every \(z\in M\) with \(z \le_{\mathcal L} x\), the map
\[
\rho_{u,v}\colon M \to M, \qquad \rho_{u,v}(t) := t\cdot u\cdot v
\]
acts as the identity on \(z\), i.e. \(\rho_{u,v}(z) = z\).
A left–right dual statement holds with left-translations and \(\mathcal R\)-below elements.
\end{lemma}

\begin{proof}
Since \(z \le_{\mathcal L} x\), there exists \(t\in M\) with \(z = t\cdot x\).
Using associativity and the relations \(x\cdot u = y\) and \(y\cdot v = x\), we compute
\[
z\cdot u\cdot v
= t\cdot x\cdot u\cdot v
= t\cdot y\cdot v
= t\cdot x
= z.\qedhere
\]
\end{proof}


\begin{lemma}[Green’s Lemma]
\label{lem:greens-lemma}
Let \(M\) be a monoid and let \(x,y\in M\) with \(x \,\mathcal R\, y\).
Fix \(u,v\in M\) such that \(x\cdot u = y\) and \(y\cdot v = x\).
Then the right-translation
\[
\rho_u \colon M \to M, \qquad \rho_u(z) := z\cdot u,
\]
restricts to a bijection from the \(\mathcal L\)-class of \(x\) onto the \(\mathcal L\)-class of \(y\);
moreover, \(\rho_v(z):=z\cdot v\) is the inverse bijection.
Additionally, these translations preserve \(\mathcal H\)-equivalence.
\end{lemma}

\begin{proof}
It suffices to verify the following:

\medskip
\noindent\emph{(1) Image in the correct \(\mathcal L\)-class.}
If \(z \,\mathcal L\, x\) then \(z\cdot u \,\mathcal L\, y\).
Indeed, writing \(z = t\cdot x\) for some \(t\), we have
\(z\cdot u = t\cdot x\cdot u = t\cdot y\), so \(z\cdot u \,\mathcal L\, y\)
(equivalently, by compatibility of \(\mathcal L\) with right-multiplication).

\medskip
\noindent\emph{(2) Injectivity on the \(\mathcal L\)-class of \(x\).}
If \(z,w \,\mathcal L\, x\) and \(z\cdot u = w\cdot u\), then \(z=w\).
By Lemma~\ref{lem:id-translation}, \(z\cdot u\cdot v = z\) and \(w\cdot u\cdot v = w\).
Hence \(z = z\cdot u\cdot v = w\cdot u\cdot v = w\).

\medskip
\noindent\emph{(3) Surjectivity onto the \(\mathcal L\)-class of \(y\).}
Let \(z \,\mathcal L\, y\). Set \(w := z\cdot v\).
Then \(w \,\mathcal L\, y\cdot v = x\), so \(w \,\mathcal L\, x\).
Moreover, by Lemma~\ref{lem:id-translation} (applied with \(z\) in the \(\mathcal L\)-class of \(y\)),
\[
z = z\cdot v\cdot u = w\cdot u = \rho_u(w),
\]
so \(z\) lies in the image of \(\rho_u\).

\medskip
\noindent\emph{(4) \(\rho_u\) and \(\rho_v\) are inverses on the respective \(\mathcal L\)-classes.}
If \(z \,\mathcal L\, x\), then \(z\cdot u\cdot v = z\) by Lemma~\ref{lem:id-translation};
if \(z \,\mathcal L\, y\), then \(z\cdot v\cdot u = z\) (the same lemma with the roles of \(x,y\) interchanged).

\medskip
\noindent\emph{(5) Preservation of \(\mathcal H\)-equivalence.}
For \(z,w \,\mathcal L\, x\) one should verify
\[
z \,\mathcal H\, w \quad\Longleftrightarrow\quad z\cdot u \,\mathcal H\, w\cdot u.
\]
Using \(\rho_{u,v}\) from Lemma~\ref{lem:id-translation} gives \(z\cdot u \,\mathcal R\, z\) and \(w\cdot u \,\mathcal R\, w\),
and the \(\mathcal L\)-compatibility from (1) supplies the \(\mathcal L\)-side; a routine transitivity argument then completes the proof. \emph{(Details omitted.)}
\end{proof}


\section{Location Theorem (Proposition 1.6)}

Throughout this section, let \(M\) be a monoid and write multiplication multiplicatively.

\begin{proposition}[Location Theorem]
\label{prop:location}
For any \(x,y\in M\), the following are equivalent:
\[
\bigl(\exists\, e\in M,\; e^2=e,\; e \,\mathcal L\, x,\; e \,\mathcal R\, y\bigr)
\quad\Longleftrightarrow\quad
(xy) \,\mathcal R\, x \ \text{ and }\ (xy) \,\mathcal L\, y.
\]
\end{proposition}

\begin{proof}
\emph{(\(\Rightarrow\)).}
Assume \(e^2=e\), \(e \,\mathcal L\, x\), and \(e \,\mathcal R\, y\).
For idempotents one has the characterizations
\[
e \,\mathcal L\, x \iff x e = x, \qquad e \,\mathcal R\, y \iff e y = y.
\]
Hence \(x e = x\) and \(e y = y\).
By multiplicative compatibility of Green’s relations (see
\ref{lem:right-mul-compat-left} and \ref{lem:left-mul-compat-right}),
from \(y \,\mathcal R\, e\) we deduce \(x y \,\mathcal R\, x e = x\),
and from \(x \,\mathcal L\, e\) we deduce \(x y \,\mathcal L\, e y = y\).

\smallskip
\emph{(\(\Leftarrow\)).}
Assume \((x y)\,\mathcal R\, x\) and \((x y)\,\mathcal L\, y\).
Consider the right-translation \(\rho_y(z)=z y\).
By Green’s Lemma (\ref{lem:greens-lemma}), \(\rho_y\) restricts to a bijection from the \(\mathcal L\)-class of \(x\)
onto the \(\mathcal L\)-class of \(x y\). Since \(y \,\mathcal L\, x y\), there exists \(t\) in the \(\mathcal L\)-class of \(x\) with
\(t y = y\).
From \((x y)\,\mathcal R\, x\) choose \(u\) with \(x y \cdot u = x\).
Applying Lemma~\ref{lem:id-translation} (with \(x \,\mathcal R\, x y\)) to \(t\) gives
\(t y u = t\). Therefore
\[
t^2 = t\cdot t = t\cdot (t y u) = (t\cdot t y)\,u = (t y)\,u = t,
\]
so \(t\) is idempotent. Moreover \(t \,\mathcal R\, y\) because \(t y = y\), and \(t \,\mathcal L\, x\) since \(t\) was chosen in the \(\mathcal L\)-class of \(x\).
Thus there exists an idempotent \(t\) with \(t \,\mathcal L\, x\) and \(t \,\mathcal R\, y\).
\end{proof}

\noindent\textbf{Remark (dual version).}
Interchanging \(\mathcal L\) and \(\mathcal R\) and using left-translation bijections yields the left–right dual location theorem.  \emph{(To be recorded separately.)}
