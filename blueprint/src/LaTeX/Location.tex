
\section{Location Theorem (Proposition 1.6)}

Throughout this section, let \(M\) be a monoid and write multiplication multiplicatively.  

\begin{proposition}[Location Theorem]
\label{prop:location}
For any \(x,y\in M\), the following are equivalent:
\[
\bigl(\exists\, e\in M,\; e^2=e,\; e \,\mathcal L\, x,\; e \,\mathcal R\, y\bigr)
\quad\Longleftrightarrow\quad
(xy) \,\mathcal R\, x \;\text{ and }\; (xy) \,\mathcal L\, y.
\]
\end{proposition}

\begin{proof}
\emph{(\(\Rightarrow\)).} Suppose there is an idempotent \(e\) with \(e \,\mathcal L\, x\) and \(e \,\mathcal R\, y\).
For idempotent \(e\), we have the characterizations
\[
\textstyle e \,\mathcal L\, x \iff x e = x, \qquad e \,\mathcal R\, y \iff e y = y.
\]
Thus \(x e = x\) and \(e y = y\).  
Green-equivalence is compatible with multiplication: from \(y \,\mathcal R\, e\) we get \(x y \,\mathcal R\, x e\), hence \(x y \,\mathcal R\, x\).  
Similarly, from \(x \,\mathcal L\, e\) we get \(x y \,\mathcal L\, e y\), hence \(x y \,\mathcal L\, y\).

\smallskip
\emph{(\(\Leftarrow\)).} Assume \((xy)\,\mathcal R\, x\) and \((xy)\,\mathcal L\, y\).
Consider the right-translation map \(\rho_y\colon M\to M,\; z\mapsto z y\).
Under \((xy)\,\mathcal R\,x\), the restriction of \(\rho_y\) to the \(\mathcal R\)-class of \(x\) is a bijection onto the \(\mathcal R\)-class of \(xy\).
Since also \((xy)\,\mathcal L\, y\), the element \(xy\) (hence \(y\)) lies in the \(\rho_y\)-image of that class.  
Thus there exists \(t\in M\) such that
\[
t y \,\mathcal R\, x y
\quad\text{and}\quad
t y \,\mathcal L\, y.
\]
From these, one extracts a witness \(u\in M\) with
\[
t y u = t \qquad\text{(a ``right-stabilizer'' identity).}
\]
Set \(e := t\). Then
\[
e^2
\;=\; e\,(t)
\;=\; e\,(t y u)
\;=\; (e y)\,u
\;=\; (t y)\,u
\;=\; t
\;=\; e,
\]
using associativity and the stabilizer identity. Hence \(e\) is idempotent.  
Moreover, from \(t y \,\mathcal L\, y\) we obtain \(e \,\mathcal R\, y\) (take the standard witnesses from the \(\mathcal L\)-equivalence), and from \(t y \,\mathcal R\, x y\) we obtain \(e \,\mathcal L\, x\). Thus there exists an idempotent \(e\) with \(e \,\mathcal L\, x\) and \(e \,\mathcal R\, y\).
\end{proof}

\noindent\textbf{Notes / Dependencies to establish (working backwards).}
\begin{itemize}
  \item \textbf{Idempotent \(\mathcal L/\mathcal R\) characterizations.}
    \begin{itemize}
      \item \(\bigl(e^2=e \ \&\  e \,\mathcal L\, x\bigr)\ \Rightarrow\ x e = x\); equivalently \(e \,\mathcal L\, x \iff x e = x\) when \(e\) is idempotent.
      \item \(\bigl(e^2=e \ \&\  e \,\mathcal R\, y\bigr)\ \Rightarrow\ e y = y\); equivalently \(e \,\mathcal R\, y \iff e y = y\) when \(e\) is idempotent.
    \end{itemize}
    (Labels to prove: \texttt{lem:L-idem-char}, \texttt{lem:R-idem-char}.)
  \item \textbf{Compatibility of Green’s equivalences with multiplication.}
    \begin{itemize}
      \item If \(a \,\mathcal R\, b\) then \(c a \,\mathcal R\, c b\) for all \(c\).
      \item If \(a \,\mathcal L\, b\) then \(a c \,\mathcal L\, b c\) for all \(c\).
    \end{itemize}
    (Labels: \texttt{lem:REquiv-mul-left}, \texttt{lem:LEquiv-mul-right}.)
  \item \textbf{Right-translation bijection on \(\mathcal R\)-classes.}  
    If \((xy)\,\mathcal R\, x\), then \(\rho_y:z\mapsto zy\) restricts to a bijection from the \(\mathcal R\)-class of \(x\) onto the \(\mathcal R\)-class of \(xy\).  
    (Label: \texttt{lem:right-translation-bijection}.)
  \item \textbf{Surjectivity toward the \(\mathcal L\)-class of \(y\).}  
    Using \((xy)\,\mathcal L\, y\) and the previous item, find \(t\) with \(t y \,\mathcal L\, y\) and \(t y \,\mathcal R\, x y\).  
    (Label: \texttt{lem:translation-hits-Lclass}.)
  \item \textbf{Right-stabilizer identity.}  
    From \(t y \,\mathcal R\, x y\) and \(t y \le_{\mathcal L} t\) (the latter follows from \(t y \,\mathcal L\, y\)), obtain \(u\) with \(t y u = t\).  
    (Label: \texttt{lem:right-id-from-REquiv-and-LLE}.)
  \item \textbf{Idempotence criterion.}  
    If \(t y u = t\) and \(t y\) is \(\mathcal R\)-equivalent to \(x y\), then \(t^2 = t\).  
    (Label: \texttt{lem:idempotence-from-stabilizer}.)
  \item \textbf{Recovering \(\mathcal L/\mathcal R\) for \(e=t\).}  
    From \(t y \,\mathcal L\, y\) build witnesses for \(t \,\mathcal R\, y\); from \(t y \,\mathcal R\, x y\) build witnesses for \(t \,\mathcal L\, x\).  
    (Labels: \texttt{lem:from-tyL-to-eRy}, \texttt{lem:from-tyR-to-eLx}.)
\end{itemize}

\noindent\textbf{Remark (Dual version).}  
The left–right dual statement also holds: exchanging \(\mathcal L\) and \(\mathcal R\), the roles of left and right multiplications, and using the analogous left-translation bijection yields the dual location theorem. \emph{(TODO: record as a separate proposition.)}
