% In this file you should put the actual content of the blueprint.
% It will be used both by the web and the print version.
% It should *not* include the \begin{document}
%
% If you want to split the blueprint content into several files then
% the current file can be a simple sequence of \input. Otherwise It
% can start with a \section or \chapter for instance.
\section{Green's preorders on a monoid}

\begin{definition}[Green's \(R\)-preorder]
\label{def:RRel}
Let \(M\) be a monoid and let \(x,y\in M\).  We write \(x \le_R y\) if there exists \(z\in M\)
such that \(x\cdot z = y\).  This defines a reflexive and transitive relation on \(M\).
\lean{Green.RRel}
\leanok
\end{definition}

\begin{lemma}[Reflexivity of \(\le_R\)]
\label{lem:RRel-refl}
For every \(x\in M\) we have \(x \le_R x\).
\lean{Green.RRel.refl}
\leanok
\uses{def:RRel}
\end{lemma}

\begin{lemma}[Transitivity of \(\le_R\)]
\label{lem:RRel-trans}
If \(x \le_R y\) and \(y \le_R z\), then \(x \le_R z\).
\lean{Green.RRel.trans}
\leanok
\uses{def:RRel}
\end{lemma}

\section{Duality via \texttt{MulOpposite}}

We consider the relation induced on the opposite monoid \(M^{\mathrm{op}}\).

\begin{lemma}[Relation on the opposite monoid]
\label{lem:RRel-op-iff}
For \(x,y\in M\), we have
\[
  \Green.RRel\;(\mathrm{op}\, x)\;(\mathrm{op}\, y)
  \iff \Green.RRel\;x\;y.
\]
In other words, the \(R\)-preorder on \(M\) corresponds to the \(R\)-preorder on 
\(\mathrm{MulOpposite}\,M\) under \(\mathrm{op}\).
\lean{RRel.op_iff}
\leanok
\uses{def:RRel, lem:RRel-refl, lem:RRel-trans}
\end{lemma}
